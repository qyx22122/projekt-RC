\documentclass[titlepage, a4paper, twoside, final]{article}

\setcounter{secnumdepth}{2} %Globina uštevilčnja naslovov
\setcounter{tocdepth}{2} %Do katere globine bodo naslovi v kazalu

\usepackage{amsmath}
\usepackage{biblatex}
\usepackage{graphicx}
\usepackage{geometry}

\graphicspath{{./slike/}}

\addbibresource{./literatura.bib}

\geometry{centering}

\begin{document}

\begin{titlepage}
  	\begin{center}
		\vspace*{1cm}
		\huge
		\textbf{Energijski izkoristek raketnega motorja na trdo gorivo}
            
		\vspace{1.5cm}

		\normalsize
		Avtorja:
		
		\large
		\textbf{Leon Smrekar Voskobojnik}

		\textbf{Tibor Maček}
		\vspace{0.5cm}
		
		\normalsize
		Mentor:
	
		\large
		\textbf{Jure Ausec}
		\vfill
		%\includegraphics{kul slika}
		\normalsize
		Raziskovalna naloga s področja ???
		\vspace{1.5cm}    

		\includegraphics{gimvic-logo}
		\vspace{0cm}

		Gimnazija Vič
		2025/2026
            
	\end{center}
\end{titlepage}

	\renewcommand{\abstractname}{Povzetek}
\begin{abstract}
	Sem gre povzetek
\end{abstract}
	\renewcommand{\abstractname}{Abstract}
\begin{abstract}
	The abstract goes here
\end{abstract}

	\renewcommand{\contentsname}{Vsebina}
	\tableofcontents

	\newpage


	\section{Uvod}
	Z razvojem raketne znanosti se je razširilo tudi amatersko raketarstvo, in z njim tudi različni tipi goriva. --neki gre se sm? yap-- Da bi prihranila na opremi, času in denarju nasploh sva si 	izbrala najenostavnejše in najcenejše gorivo. --neki gre tut sm-- Kaj kmalu pa se nama je porodilo vprašanje učinkovitosti in tako so nastali temelji za raziskovalno nalogo. Ker pa mnogi, 	vključno z nama, nimajo oziroma ne želijo porabiti velikih vsot za "igranje" z oksidativnimi snovmi sva se omejila in poizkusila narediti motor iz prostodostopnih in nizkocenovnih 					materialov, kar nama je tudi uspelo.%(foreshadowing)

	\subsection{Pod uvod}

	bla 2.0

	\section{Teoretičen del}
	Raketni motorji delujejo, zaradi zakona o ohranitvi gibalne količine; motor pospeši vroče pline v nasprotno smer gibanja in s tem poveča hitrost rakete.
	Pri raketnih motorjih na trdno grivo pride ta plin iz reakcije med gorivom in oksidatorjem, ki sta oba v trdnem agregatnem stanju. V zgorevalni komori se zaradi tega ustvari tlak.
	Vroči plini uhajajo skozi šobo, ki jih zaradi zožanja v vratu, pospeši do zvočne hitrosti in potem ko se začne tlak manjšati do normalnega zračnega tlaka pospešijo tudi preko zvočne 				hitrosti. 
	Funkcija šobe je izkoristiti plinov tlak in temperaturo, da ga pospeši do visokih hitrosti in mu s tem poveča gibalno količino.
	\subsection{Specifičen impulz}
	Impuloz ali sunek sile je fizikalna količina, ki spremeni gibalno količino telesa, ki se giba. Izračunamo ga kot spremembo gibalne količine.
\begin{equation}
	I = G_1 - G_0
\end{equation}
	Iz impulza izpeljemo količino za učinkovitost raketnih motorjev, ki jo imenujemo specifičen impulz. Ta nam pove razmerje med impulzom motorja in maso goriva. \cite{wiki:Specific_impulse2025}
\begin{equation}
	I_{SP} = \frac{I}{m \cdot g_0}
\end{equation}


	\section{Empirični del}

	\section{Rezultati}

	\section{Razprava}

	\section{Zaključek}


	\printbibliography
	[
	heading=bibintoc,
	title={Literatura}
	]

\end{document}
