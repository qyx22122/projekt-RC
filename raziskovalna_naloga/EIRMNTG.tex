\documentclass[titlepage, a4paper, twoside, final]{article}

\setcounter{secnumdepth}{2} %Globina uštevilčnja naslovov
\setcounter{tocdepth}{2} %Do katere globine bodo naslovi v kazalu

\usepackage{amsmath}
\usepackage{mhchem}
\usepackage[backend=biber, date=iso]{biblatex}
\usepackage{graphicx}
\usepackage{geometry}

%Formatiranje datuma
\DeclareFieldFormat{urldate}{ dostopano \DTMdisplaydate{\thefield{urlyear}}{\thefield{urlmonth}}{\thefield{urlday}}{}}


\graphicspath{{./slike/}}

\addbibresource{./literatura.bib}

\geometry{centering}

\begin{document}

\begin{titlepage}
  	\begin{center}
		\vspace*{1cm}
		\huge
		\textbf{Energijski izkoristek raketnega motorja na trdo gorivo}
            
		\vspace{1.5cm}

		\normalsize
		Avtorja:
		
		\large
		\textbf{Leon Smrekar Voskobojnik}

		\textbf{Tibor Maček}
		\vspace{0.5cm}
		
		\normalsize
		Mentor:
	
		\large
		\textbf{Jure Ausec}
		\vfill
		%\includegraphics{kul slika}
		\normalsize
		Raziskovalna naloga s področja ???
		\vspace{1.5cm}    

		\includegraphics{gimvic-logo}
		\vspace{0cm}

		Gimnazija Vič
		2025/2026
            
	\end{center}
\end{titlepage}

	\renewcommand{\abstractname}{Povzetek}
\begin{abstract}
	Sem gre povzetek
\end{abstract}
	\renewcommand{\abstractname}{Abstract}
\begin{abstract}
	The abstract goes here
\end{abstract}

	\renewcommand{\contentsname}{Vsebina}
	\tableofcontents

	\newpage


	\section{Uvod}
	Z razvojem raketne znanosti se je razširilo tudi amatersko raketarstvo, in z njim tudi različni tipi goriva. --neki gre se sm? yap-- Da bi prihranila na opremi, času in denarju nasploh sva si 	izbrala najenostavnejše in najcenejše gorivo. --neki gre tut sm-- Kaj kmalu pa se nama je porodilo vprašanje učinkovitosti in tako so nastali temelji za raziskovalno nalogo. Ker pa mnogi, 	vključno z nama, nimajo oziroma ne želijo porabiti velikih vsot za "igranje" z oksidativnimi snovmi sva se omejila in poizkusila narediti motor iz prostodostopnih in nizkocenovnih 					materialov, kar nama je tudi uspelo.%(foreshadowing)

	\subsection{Pod uvod}

	bla 2.0

	\section{Teoretičen del}
	Raketni motorji delujejo zaradi zakona o ohranitvi gibalne količine; motor pospeši vroče pline v nasprotno smer gibanja in s tem poveča hitrost rakete.
	Pri raketnih motorjih na trdno grivo ta plin pride iz reakcije goriva in oksidatorja, obeh v trdnem agregatnem stanju. V zgorevalni komori je zaradi tega ustvarjen tlak.
	Vroči plini uhajajo skozi šobo, ki jih zaradi zožanja v grlu pospeši do zvočne hitrosti. Ko se tlak manjša do normalnega zračnega tlaka plini pospešujejo tudi preko zvočne
	hitrosti.
	Funkcija šobe je izkoriščanje tlaka in temperature plinov za pospeševanje do visokih hitrosti in s tem povečevanje gibalne količine.
	\subsection{Specifičen impulz}
	Impulz ali sunek sile je fizikalna količina, ki spremenja gibalno količino telesa, ki se giba. Izračunamo ga kot spremembo gibalne količine.
\begin{equation}
	I = G_1 - G_0
\end{equation}
	Iz impulza izpeljemo količino za učinkovitost raketnih motorjev, ki jo imenujemo specifičen impulz. Ta nam pove razmerje med impulzom motorja in maso goriva. \cite{wiki:Specific_impulse2025}
\begin{equation}
	I_{SP} = \frac{I}{m \cdot g_0}
\end{equation}
	Tej količini sva v raziskovalni nalogi posvetila največ pozornosti.

	\subsection{Gorivo}
	Vrst trdnih goriv za raketne motorje je zelo veliko, ampak imajo vsa v osnovi podobno sestavo - mešanico goriva in oksidatorja. V večini primerov so dodani še
	katalizatorji in sredstva za strjevanje. Midva sva se odločila za gorivo, ki ga pogosto poimenujejo "rocket candy". Sestavljen je iz saharoze in kalijevega nitrata ter občasno
	tudi nekaterih primesi (npr. \ce{Fe2O3}). Izbrala sva razmerje saharoze in \ce{KNO3} 35/65 \cite{nakkaKNSU2025}.



	\section{Empirični del}

	\section{Rezultati}

	\section{Razprava}

	\section{Zaključek}


	\printbibliography
	[
	heading=bibintoc,
	title={Literatura}
	]

\end{document}
