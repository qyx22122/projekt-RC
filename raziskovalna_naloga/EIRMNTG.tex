\documentclass[titlepage, a4paper, twoside, draft]{article}

%\setcounter{secnumdepth}{2} %Globina uštevilčnja naslovov
%\setcounter{tocdepth}{2} %Do katere globine bodo naslovi v kazalu

\begin{document}

\title{Energijski izkoristek raketnega motorja na trdo gorivo}
\author{Leon Smrekar Voskobojnik in Tibor Maček}
\date{3.12.2025}
\maketitle

\renewcommand{\abstractname}{Povzetek}
\begin{abstract}
Sem gre povzetek
\end{abstract}
\renewcommand{\abstractname}{Abstract}
\begin{abstract}
The abstract goes here
\end{abstract}

\renewcommand{\contentsname}{Vsebina}
\tableofcontents

\newpage


\section{Uvod}

Z razvojem raketne znanosti se je razširilo tudi amatersko raketarstvo, in z njim tudi različni tipi goriva. --neki gre se sm? yap-- Da bi prihranila na opremi, času in denarju nasploh sva si izbrala najenostavnejše in najcenejše gorivo. --neki gre tut sm-- Kaj kmalu pa se nama je porodilo vprašanje učinkovitosti in tako so nastali temelji za raziskovalno nalogo. Ker pa mnogi, vključno z nama, nimajo oziroma ne želijo porabiti velikih vsot za "igranje" z oksidativnimi snovmi sva se omejila in poizkusila narediti motor iz prostodostopnih in nizkocenovnih materialov, kar nama je tudi uspelo.%(foreshadowing)

\subsection{Pod uvod}

bla 2.0


\end{document}