\documentclass[titlepage, a4paper, twoside, draft]{article}

\setcounter{secnumdepth}{2} %Globina uštevilčnja naslovov
\setcounter{tocdepth}{2} %Do katere globine bodo naslovi v kazalu

\usepackage{amsmath}
\usepackage{biblatex}
\addbibresource{./literatura.bib}

\begin{document}

\title{Energijski izkoristek raketnega motorja na trdo gorivo}
\author{Leon Smrekar Voskobojnik in Tibor Maček}
\date{3.12.2025}
\maketitle

\renewcommand{\abstractname}{Povzetek}
\begin{abstract}
Sem gre povzetek
\end{abstract}
\renewcommand{\abstractname}{Abstract}
\begin{abstract}
The abstract goes here
\end{abstract}

\renewcommand{\contentsname}{Vsebina}
\tableofcontents
\clearpage



\section{Uvod}

Bla bla bla
\cite{wiki:SPR2025}

\subsection{Pod uvod}

bla 2.0

\section{Teoretičen del}
Raketni motorji delujejo, zaradi zakona o ohranitvi gibalne količine; motor pospeši vroče pline v nasprotno smer gibanja in s tem poveča hitrost rakete.
Pri raketnih motorjih na trdno grivo pride ta plin iz reakcije med gorivom in oksidatorjem, ki sta oba v trdnem agregatnem stanju. V zgorevalni komori se zaradi tega ustvari tlak.
Vroči plini uhajajo skozi šobo, ki jih zaradi zožanja v vratu, pospeši do zvočne hitrosti in potem ko se začne tlak manjšati do normalnega zračnega tlaka pospešijo tudi preko zvočne hitrosti. 
Funkcija šobe je izkoristiti plinov tlak in temperaturo, da ga pospeši do visokih hitrosti in mu s tem poveča gibalno količino.
\subsection{Specifičen impulz}
Impuloz ali sunek sile je fizikalna količina, ki spremeni gibalno količino telesa, ki se giba. Izračunamo ga kot spremembo gibalne količine.
\begin{equation}
	I = G_1 - G_0
\end{equation}
Iz impulza izpeljemo količino za učinkovitost raketnih motorjev, ki jo imenujemo specifičen impulz. Ta nam pove razmerje med impulzom motorja in maso goriva. 
\begin{equation}
	I_{SP} = \frac{I}{m \cdot g_0}
\end{equation}


\section{Empirični del}

\section{Rezultati}

\section{Razprava}

\section{Zaključek}


\printbibliography[
heading=bibintoc,
title={Literatura}
]

\end{document}
